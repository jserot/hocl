%%%%%%%%%%%%%%%%%%%%%%%%%%%%%%%%%%%%%%%%%%%%%%%%%%%%%%%%%%%%%%%%%%%%%%%%%%%%%%%%%%%%%%%%%
%%                                                                                     %%
%%                This file is part of the CAPH Compiler distribution                  %%
%%                            http:%/caph.univ-bpclermont.fr                           %%
%%                                                                                     %%
%%                                  Jocelyn SEROT                                      %%
%%                         Jocelyn.Serot@univ-bpclermont.fr                            %%
%%                                                                                     %%
%%         Copyright 2011-2018 Jocelyn SEROT.  All rights reserved.                    %%
%%  This file is distributed under the terms of the GNU Library General Public License %%
%%      with the special exception on linking described in file ..%LICENSE.            %%
%%                                                                                     %%
%%%%%%%%%%%%%%%%%%%%%%%%%%%%%%%%%%%%%%%%%%%%%%%%%%%%%%%%%%%%%%%%%%%%%%%%%%%%%%%%%%%%%%%%%

\chapter{Static semantics}
\label{chap:static}

%\newcommand{\sy}[1]{\mathbf{#1}}
%\newcommand{\sem}[1]{\mathsf{#1}}
%\newcommand{\tm}[1]{\mathtt{#1}}
%\newcommand{\ut}[1]{\emph{#1}}
%\newcommand{\cat}[1]{\text{#1}}
 \newcommand{\semwild}{.}


The static semantics of \caph programs is in the form of a set of \emph{boxes} interconnected by
\emph{wires}. Boxes result from the instanciation of actors and wires from the data
dependencies expressed in the definition section.

\medskip
The static semantics is built upon the semantic domain \emph{SemVal} given below.

In the semantic rules, tuples will be denoted $\langle x_1, \ldots, x_n \rangle$.  The size of a
tuple $t$ will be denoted $|t|$.  
The notation $D^*$ is used to define the domain of all tuples of $D$:
$\langle\rangle,~ \langle D\rangle,~ \langle D,D \rangle,~ \ldots$
The notation $D^+$ is used to define the domain of all non-empty tuples of $D$:
$\langle D\rangle,~ \langle D,D \rangle,~ \ldots$
A single value and a tuple of size one will be considered as semantically equivalent (this
simplifies the description of the rule dealing with the instanciation of actors).
We will use the notation $.$ to denote the "\emph{dont care}" (wildcard)
value in rules (in order to avoid confusion with the "\emph{dont care}" value at the syntactical level (\_).

If $E$ is an environment, and $f$ a function, then we denote by $\Pi_f(E)$ the environment obtained by
applying $f$ to each element of $\text{codom}(E)$ : $\Pi_f(E) = \lbrace x \mapsto f(E(x)), x \in
dom(E) \rbrace$. If $F$ is a set $\lbrace f_1, \ldots, f_n \rbrace$ of functions, then we abbreviate
$\Pi_F(E)$ the successive applications $\Pi_{f_1}(\Pi_{f_2}(\ldots \Pi_{f_n}(E)\ldots))$.

\begin{equation*}
  \renewcommand{\arraystretch}{1}
  \begin{array}[t]{c|c|l|l}
    \textbf{Variable}  & \textbf{Set ranged over}  &  \textbf{Definition} & \textbf{Meaning} \\\hline 
    \text{v}           & \sem{EVal}  & \sem{Int} + \sem{Bool} + \sem{ETuple} & \\
    &             & + \sem{ECon} + \sem{EPrim} & \\
    &             & + \sem{EFun} + \sem{Unknown} & \text{Expression-level semantic value} \\
    \iota              & \sem{Int}   & \lbrace \ldots, -2, -1, 0, 1, \ldots \rbrace & \\
    \beta              & \sem{Bool}  & \lbrace \text{true}, \text{false} \rbrace    & \text{Builtin constant} \\
    \text{c}           & \sem{ECon}  & \txt{id}~\sem{EVal}^* & \text{Constructed value} \\
    \text{vs}          & \sem{ETuple} & \sem{EVal}^{+} & \text{Expression-level Tuple} \\
    \EE                & \sem{EEnv}   & \lbrace \txt{id} \mapsto \sem{EVal} \rbrace & \text{Expression-level environment} \\
    \rho               & \sem{NVal} & \sem{Loc} + \sem{Act} + \sem{Clos} + \sem{Tuple} & \\
    &                 & + \sem{Box} + \sem{Wire} + \sem{NEVal} &  \text{Network-level semantic value} \\
    \NE                & \sem{NEnv}   & \lbrace \txt{id} \mapsto \sem{NVal} \rbrace & \text{Network-level environment} \\
    \rho s             & \sem{NTuple}  & \sem{NVal}^{+} & \text{Network-level tuples} \\
    \text{cl}          & \sem{Clos}   & \tttuple{\emtxt{n\_pattern}}{\emtxt{n\_expr}}{\sem{NEnv}} & \text{Network-level closure} \\
    \text{a}           & \sem{Act} & \langle \sem{id},~\sem{id}^*, ~\sem{id}^*,~\sem{id}^*, ~\lbrace
    \txt{id} \mapsto \emtxt{expr}
    \rbrace, ~\emtxt{rules}\rangle & \text{Static actor description} \\
    \ell             & \sem{Loc} & \ttuple{\sem{bid}}{\sem{sel}} & \text{Location} \\
    \text{b}           & \sem{Box} & \langle \sem{id}, ~\sem{btag},~ \{\sem{id}\mapsto\sem{EVal}\},~ \{\sem{id}\mapsto\sem{EVal}\} & \text{Box} \\
    &           &         \{\sem{id}\mapsto\sem{wid}\},~\{\sem{id}\mapsto\sem{wid}\},~\emtxt{rules}) & \\
    \text{t}          & \sem{btag} & \sem{BBox} + \sem{BIn} + \sem{BOut} + \sem{BDummy} & \text{Box tag} \\
    \text{w}          & \sem{Wire} & \langle\langle\sem{bid},\sem{sel}\rangle,\langle\sem{bid},\sem{sel}\rangle\rangle & \text{Wire} \\
    \rho'             & \sem{NEVal}  & \sem{Int} + \sem{Bool} & \\
    \BB               &              & \lbrace \sem{bid} \mapsto \sem{Box} \rbrace & \text{Box environment} \\
    \WW               &              & \lbrace \sem{wid} \mapsto \sem{Wire} \rbrace & \text{Wire environment} \\
    \text{l},\text{l'} & \sem{bid} & \lbrace 1, 2, \ldots \rbrace & \text{Box id} \\
    \text{k},\text{k'} & \sem{wid} & \lbrace 1, 2, \ldots \rbrace & \text{Wire id} \\
    \text{s},\text{s'} & \sem{sel} & \lbrace 1, 2, \ldots \rbrace & \text{Slot indexes} \\
\end{array}
\end{equation*}

\medskip The semantic categories $\sem{EVal}$ (for expression-level values), $\sem{Tuple}$ and
$\sem{Clos}$ are classical. The semantic category $\sem{NEVal}$ is used to wrap a subset of
expression-level semantic values (integer and boolean constants) as network values. This is required
to be able to manipulate actor parameters at the network level.
% $\sem{EExtFun}$ values describe external functions.
$\sem{Unknown}$ values are used for uninitialized actor variables.

\medskip
$\sem{Act}$ values describe actors. An actor is a 6-tuple consisting of 
\begin{itemize}
\item a name,
\item a list of parameter names,
\item a list of input and output names,
\item a list of local variables, with an optional expression describing the initial value,
\item a list of rules.
\end{itemize}

Note that initial value of actor variables is described as an \emph{unevaluated} expression since
the corresponding semantic value can only be computed when the actor is instanciated as a box (it
may depend on the actual value of the actor parameters, set at the network level).

The actor rules are just copied from the abstract syntaxt representation, since the static semantics
does not deal with the \emph{behavior} of actors (actors are essentially viewed as ``black boxes''
at this level). They will be used by the dynamic semantics.

\medskip
$\sem{Box}$ values describe boxes (instanciated actors). A box is a 7-tuple consisting of 
\begin{itemize}
\item the name of corresponding actor,
\item a tag, for distinguishing ``ordinary'' boxes (resulting from the instanciation of actors) from
  boxes corresponding to i/o streams\footnote{The tag $\sem{BDummy}$ is used internally for handling
    recursive definitions.},
\item a list of parameters, with their actual values,
\item a list of local variables, with their actual values,
\item a list of inputs and outputs slots, each associating a name with a wire index,
\item a list of rules (copied verbatim from the actor description). 
\end{itemize}

\medskip
$\sem{Wire}$ values are used to describe
interconnexions between boxes. Each box has a unique index ($\sem{bid}$). A \emph{location}
($\sem{Loc}$) identifies a given i/o slot of a box (inputs and outputs are numbered from $1$ to
$n$). So wires are actually described by a pair of locations.

\medskip
Semantic rules are given in a context consisting of
\begin{itemize}
\item a \emph{type environment} $\TE$ recording type constructors\footnote{This environment is
    required for evaluating type coercion operations.},
\item a \emph{expression-level environment} $\EE$, mapping identifiers to expression-level semantic values,
\item a \emph{network-level environment} $\NE$, mapping identifiers to network-level semantic values,
\item a \emph{box environment} $\BB$, mapping indexes to \emph{boxes},
\item a \emph{wire environment} $\WW$, mapping indexes to \emph{wires}.
\end{itemize}

The notations for accessing and manipulating environments are similar to those defined in
chapter~\ref{chap:typing}. Empty environments are noted $\emptyenv$.

\newpage
\section{Programs}
\label{sec:static-programs}

\ruleheader{\TE_0,\EE_0 \vdash \text{Program} \gives \TE,\EE,\NE,\BB,\WW}

\infrule[Program]
%{ \TE_0 \vdash \emtxt{tydecls} \gives \TE \\
{ \TE, \EE \vdash \emtxt{valdecls} \gives \EE' \\
  \TE, \EE' \vdash \emtxt{actdecls} \gives \NE_a \\
  \TE \vdash \emtxt{strdecls} \gives \NE_s,\BB \\
  \TE, \EE \oplus \EE', \NE_a \oplus \NE_s, \BB \vdash \emtxt{netdecls} \gives \BB',\WW' \\
  \WW' \vdash \BB \oplus \BB' \gives \BB''}
{\TE, \EE \vdash
 % \sy{program}~\emtxt{tydecls}~\emtxt{valdecls}~\emtxt{actdecls}~\emtxt{strdecls}~\emtxt{netdecls}~\gives~\TE,\EE,\NE_a,\BB'',\WW'}
  \sy{program}~\emtxt{valdecls}~\emtxt{actdecls}~\emtxt{strdecls}~\emtxt{netdecls}~\gives~\TE,\EE',\NE_a,\BB'',\WW'}

The initial environment $\EE$ contains the internal value\footnote{Function from
  semantic values to semantic values.} of the expression-level builtin primitives (\texttt{+},
\texttt{-}, \ldots). 

\medskip
The result consists of
\begin{itemize}
\item a type environment, containing the declared type constructors\footnote{These constructors are
    required to evaluate \emph{cast} expressions.},
\item an expression-level environment, containing the declared constants and functions,
\item a network-level environment, containing the declared actors.
\item a indexed set of boxes and wires, describing the network,
\end{itemize}

\medskip
Evaluation of network-level definitions produces a set of boxes and a set of wires, in which the latter
refer to the former. The last subgoal of the previous rule -- detailed in the following rules --
makes boxes also refer to the wires (a kind of  ``reverse-wiring'' step).

\medskip
\ruleheader{\WW \vdash \BB \gives \BB'}

\infrule[]
{\forallin{i}{1}{n},~~ \WW \vdash \emtxt{b}_i \gives \emtxt{b'}_i}
{\WW \vdash \{ \emtxt{b}_1,~ \ldots,~ \emtxt{b}_n \} ~\gives \{ \emtxt{b'}_1,~ \ldots,~ \emtxt{b'}_n \}} 

\ruleheader{\WW \vdash \txt{b} \gives \txt{b}'}
\infrule[]
{\WW \vdash \txt{bid},\emtxt{bins} \gives \emtxt{bins}' \andalso \WW \vdash \txt{bid},\emtxt{bouts} \gives \emtxt{bouts}'}
{\WW \vdash \txt{bid} \mapsto
  \tttttuple{\emtxt{id}}{\emtxt{tag}}{\emtxt{bparams}}{\emtxt{bins}}{\emtxt{bouts}}~
\gives \txt{bid} \mapsto \tttttuple{\emtxt{id}}{\emtxt{tag}}{\emtxt{bparams}}{\emtxt{bins'}}{\emtxt{bouts'}}}

\ruleheader{\WW \vdash \txt{bid},\emtxt{bios} \gives \emtxt{bios}'}

\infrule[]
{\forallin{i}{1}{n},~~ \WW \vdash \txt{bid}, i \gives \txt{wid}_i}
{\WW \vdash \txt{bid},~ \{ \txt{id}_1 \mapsto 0,~ \ldots,~ \txt{id}_n \mapsto 0 \} ~\gives \{ \txt{id}_1 \mapsto
  \txt{wid}_1,~ \ldots,~ \txt{id}_n \mapsto \txt{wid}_n \}} 

\ruleheader{\WW \vdash \txt{bid},\txt{sel} \gives \txt{wid}}

\infrule[]
{\WW(k) = \ttuple{\semwild}{\ttuple{\txt{bid}}{\txt{sel}}}}
{\WW \vdash \txt{bid},\txt{sel} \gives k}

% \ifthenelse{\boolean{devel}}{
% \section{Type declarations}
% \label{sec:static-type-declarations}

% \tbw
% }{}

\section{Value declarations}
\label{sec:static-value-declarations}

\ruleheader{\TE, \EE \vdash \text{ValDecls} \gives \EE'}

\infrule[ValDecls]
{\forallin{i}{1}{n},~~ \TE,\EE_{i-1} \vdash \emtxt{valdecl}_i \gives \EE_i,~~ \EE_0=\emptyenv}
{\TE \vdash \emtxt{valdecl}_1~\ldots~\emtxt{valdecl}_n \gives \EE_n}

A value definition can only refer to a value defined before (hence the order of declaration is relevant).

\infrule[ConstDecl]
{\TE,\EE \vdash \emtxt{exp} \gives v ~~ \emtxt{is\_static\_const}(v)}
{\TE,\EE \vdash \tm{const}~\txt{id}~\txt{=}~\emtxt{exp} \gives [\txt{id} \mapsto v]}

The predicate \emph{is\_static\_const} is true only for integer and boolean constants.
% and arrays of constants.

\infrule[FunDecl]
{}
{\TE,\EE \vdash \tm{fun}~\txt{id}~\txt{=}~\emtxt{pat}~\rightarrow~\emtxt{exp} \gives [\txt{id} \mapsto
  \sem{EFun(\emtxt{pat},\emtxt{exp})}]}

% \infrule[ExtFunDecl]
% {}
% {\TE,\EE \vdash \tm{extern}~\txt{id}~\txt{:}~\emtxt{ty} \gives [\txt{id} \mapsto
%   \sem{ExtFun}]}

\section{Expressions}
\label{sec:static-expressions}

This set of rules is classical.
Note that the type environment $\TE$ is needed to process \emph{cast} expressions.

\ruleheader{\TE,\EE \vdash \sem{Exp} \gives \txt{v}}

\infrule[EConst]
{}
{\TE,\EE \vdash \text{int}/\text{bool} \gives \sem{Int}/\sem{Bool}}

\todo{Should we include arrays consts in the previous rule ?}
\todo{Clarify notation : where is the actual \emph{value} in the previous rule ?!}

\infrule[EVar]
{\EE(\txt{id}) = \vv}
{\TE,\EE \vdash \txt{id} \gives \vv}

\infrule[ECon]
{\forallin{i}{1}{n}, ~~\TE,\EE \vdash \emtxt{exp}_i \gives \vv_i}
{\TE,\EE \vdash \txt{con}~\tuplen{\emtxt{exp}} \gives \sem{ECon}~\tuplen{\vv}}

\infrule[EFunApp1]
{\EE(\txt{id}) = \sem{EPrim}~f \\
 \forallin{i}{1}{n}, ~~\TE,\EE \vdash \emtxt{exp}_i \gives \vv_i \\
 \overline{f}\parenn{\vv} = \vv}
{\TE,\EE \vdash \txt{id}~\mathbf{(}\seqn{\emtxt{exp}}\mathbf{)} \gives \vv}

Rule \textsc{EFunApp1} deals with the application of builtin functions. 
For each function $f$, we assume that $\overline{f}(v_1,\dots,v_k)$ denotes the proper
resulting value, provided that $k$ is equal to the function arity and
that the arguments $v_1,\dots,v_k$ are of the proper type\footnote{This is ensured by the
  the type cheking stage.}.

\infrule[EFunApp2]
{\EE(\txt{id}) = \sem{EFun}~\ttuple{\tuplen{\txt{id}}}{\emtxt{exp}} \\
 \forallin{i}{1}{n}, ~~\TE,\EE \vdash \emtxt{exp}_i \gives \vv_i \\
\TE, \EE \oplus [\txt{id}_1 \mapsto \vv_1, \ldots, \txt{id}_n \mapsto \vv_n ] \vdash \emtxt{exp} \gives \vv}
{\TE,\EE \vdash \txt{id}~\mathbf{(}\seqn{\emtxt{exp}}\mathbf{)} \gives \vv}

Rule \textsc{EFunApp2} deals with the application of globally defined functions. It follows the classical call-by-value
strategy (the function expression is evaluated in an environment augmented with the bindings resulting
from binding its pattern to the argument). 

\infrule[ELet]
{\TE,\EE \vdash \emtxt{exp}_2 \gives \vv \andalso \TE,\EE \oplus [\txt{id} \mapsto \vv] \vdash \emtxt{exp}_1 \gives \vv'}
{\TE,\EE \vdash \tm{let}~ \txt{id} \tm{=}~ \emtxt{exp}_2~ \tm{in}~ \emtxt{exp}_1 \gives \vv'}

\infrule[ECond0]
{\TE,\EE \vdash \emtxt{exp} \gives \sem{true} \andalso \TE,\EE \vdash \emtxt{exp}_1 \gives \vv}
{\TE,\EE \vdash \tm{if}~ \emtxt{exp}~ \tm{then}~ \emtxt{exp}_1~ \tm{else}~ \emtxt{exp}_2 \gives \vv}

\infrule[ECond1]
{\TE,\EE \vdash \emtxt{exp} \gives \sem{false} \andalso \TE,\EE \vdash \emtxt{exp}_2 \gives \vv}
{\TE,\EE \vdash \tm{if}~ \emtxt{exp}~ \tm{then}~ \emtxt{exp}_1~ \tm{else}~ \emtxt{exp}_2 \gives \vv}

% \infrule[EArrRd]
% {\EE(\txt{id}) = \sem{Array}~\tuplen{\vv} \andalso \TE,\EE \vdash \emtxt{exp} \gives \sem{Int}~i}
% {\TE,\EE \vdash \txt{id}~\tm{[}\emtxt{exp}\tm{]} \gives \vv_i}

% \infrule[EArrUpd]
% {\EE(\txt{id}) = \sem{Array}~\tuplen{\vv} \andalso \TE,\EE \vdash \emtxt{exp}_1 \gives \sem{Int}~i
% \andalso \TE,\EE \vdash \emtxt{exp}_2 \gives \vv'}
% {\TE,EE \vdash \txt{id}~\tm{[}\emtxt{exp}_1~\tm{\leftarrow}~\emtxt{exp}_2\tm{]} \gives
%  \vv[i\leftarrow \vv']}

% In the above rule, we note $t[i \leftarrow v]$ the array $t'$ defined by $t'[j]=t[j]$ for all
% $j\not=i$ and $t'[i]=t[i]$.

\infrule[ECast]
{\TE,\EE \vdash \emtxt{exp} \gives \vv \andalso \TE \vdash \emtxt{ty} \gives \tau \TE \vdash
  \emtxt{coerce}(\vv,\tau)=\vv'}
{\TE,\EE \vdash \emtxt{exp}~ \tm{:}~ \emtxt{ty} \gives \vv'}

The function \emph{coerce} coerces a value to a given type. Coercibility has been checked by the at
the typing stage. The coercibility relation and the behavior of the \emph{coerce} function have been
defined in Sec.~\ref{sec:e-coerce}.

\section{Actor declarations}
\label{sec:static-actor-declarations}

\ruleheader{\TE,\EE \vdash \text{ActorDecls} \gives \NE}

\infrule[ActorDecls]
{\forallin{i}{1}{n},~~ \TE,\EE \vdash \emtxt{actdecl}_i \gives \NE_i}
{\TE,\EE \vdash \emtxt{actdecl}_1~\ldots~\emtxt{actdecl}_n \gives \C{i=1}{n}{\NE_i}}

\infrule[ActorDecl]
{\vdash \text{params} \gives \text{params}'\\
 \vdash \text{ins} \gives \text{ins}'\\
 \vdash \text{outs} \gives \text{outs}' \\
 \vdash \text{vars} \gives \text{vars}'}
{\TE,\EE \vdash
  \actorr~\txt{id}~\txt{params}~\txt{ins}~\txt{outs}~{\txt{vars}}~{\txt{rules}}
  \\ \gives [\txt{id} \mapsto
  \sem{Act}~\ttttttuple{\txt{id}}{\txt{params}'}{\txt{ins}'}{\txt{outs}'}{\txt{vars}'}{\txt{rules}}]}

The rule \textsc{ActorDecl} builds $\sem{Act}$ semantic values from the corresponding description at the
abstract syntax level.

\medskip
\ruleheader{\vdash \txt{params}/\txt{ins}/\txt{outs} \gives \txt{params}'/\txt{ins}'/\txt{outs}'}

\infrule[ActorParamsInsOuts]
{}
{\vdash \txt{id}_1 : \ty_1 ~\ldots~\txt{id}_n : \ty_n \gives \{ \txt{id}_1, \ldots, \txt{id}_n \}}

\medskip
\ruleheader{\vdash \txt{vars} \gives \txt{vars}'}

\infrule[ActorVars]
{\forallin{i}{1}{n},~~ \vdash \emtxt{var}_i \gives \EE_i}
{\vdash \emtxt{var}_1~\ldots~\emtxt{var}_n \gives \C{i=1}{n}{\EE_i}}

\ruleheader{\vdash \txt{var} \gives \txt{var}'}

% \infrule[ActVar]
% {}
% {\TE,\EE~\vdash \txt{id} ~:~ \ty \gives [\txt{id} \mapsto \sem{Unknown}]}

\infrule[ActVar]
{}
{\vdash \txt{id} ~:~ \ty ~=~ \emtxt{exp} \gives [\txt{id} \mapsto \emtxt{exp}]}

\section{Stream declarations}
\label{sec:static-stream-declarations}

\ruleheader{\TE \vdash \text{StreamDecls} \gives \NE, \BB}

\infrule[StreamDecls]
{\forallin{i}{1}{n},~~ \vdash \emtxt{strdecl}_i \gives \NE_i,\BB_i}
{\TE \vdash \emtxt{strdecl}_1~\ldots~\emtxt{strdecl}_n \gives \C{i=1}{n}{\NE_i},\C{i=1}{n}{\BB_i}}

\infrule[InStreamDecl]
{\text{b}=\langle\txt{id},\sem{BIn},\emptytuple,\emptytuple,\tuple{"o"\mapsto 0}\rangle \andalso
  \txt{l}=\textsc{NewBid}()}
{\TE \vdash \tm{stream}~\txt{id}~\ty~\txt{from}~\txt{id'} \gives [\txt{id} \mapsto \sem{Loc}(l,1)], [l
  \mapsto \txt{b}]}

\infrule[OutStreamDecl]
{\text{b}=\langle\txt{id},\sem{BOut},\emptytuple,\tuple{"i" \mapsto 0},\emptytuple\rangle \andalso
  \txt{l}=\textsc{NewBid}()}
{\TE \vdash \tm{stream}~\txt{id}~\ty~\txt{to}~\txt{id'} \gives [\txt{id} \mapsto \sem{Loc}(l,1)], [l
  \mapsto \txt{b}]}

\medskip Each stream declaration creates a new box and enters the corresponding location in the
environment.  The $\sem{BIn}$ (resp. $\sem{BOut}$) boxes have no parameter, no input (resp. output).
Wire identifiers in the input (resp.  output) list are set to 0 at
this stage. They will be updated when all definitions have been processed (see last sub-goal of rule
\textsc{Program}).

The function $\textsc{NewBid}$ returns a new, fresh box index (i.e. an index $l$ such as $l \not\in Dom(\BB)$).

\section{Network declaration}
\label{sec:static-network-declarations}

This section only deals with \emph{non recursive} declarations. Recursive declarations are handled
in Sec.~\ref{sec:recurs-netw-decl}.

\ruleheader{\TE,\EE,\NE \vdash \text{NetDecls} \gives \NE',\BB,\WW}

\infrule[NetDecls]
{\forallin{i}{1}{n}, ~~\TE,\EE,\NE_{i-1},\BB_{i-1},\WW_{i-1} \vdash \emtxt{netdecl}_i \gives \NE_i, \BB_i, \WW_i \andalso \NE_0=\NE}
{\TE,\EE,\NE \vdash \emtxt{netdecl}_1,\ldots,\emtxt{netdecl}_n \gives \NE_n,~\BB_n,~\WW_n}

\ruleheader{\TE,\EE,\NE,\BB,\WW \vdash \text{NetDecl} \gives \NE',\BB',\WW'}

\infrule[NetDecl]
{\forallin{i}{1}{n},~~ \TE,\EE,\NE,\BB,\WW \vdash \ibinding{npat}{nexp}{i} \gives \NE_i,\BB_i,\WW_i \\
\NE' = \C{i=1}{n}{\NE_i},~ \BB' = \C{i=1}{n}{\BB_i},~ \WW' = \C{i=1}{n}{\WW_i}}
{\TE,\EE,\NE,\BB,\WW \vdash \mathbf{net}~ \bindings{npat}{nexpr}{n} \gives \NE \oplus \NE', \BB \oplus \BB', \WW \oplus
  \WW'}

\ruleheader{\TE,\EE,\NE,\BB,\WW \vdash \bindings{npat}{nexpr}{n} \gives \NE',\BB',\WW'}

\infrule[NetBindings]
{\forallin{i}{1}{n},~~ \TE,\EE,\NE,\BB,\WW \vdash \ibinding{npat}{nexp}{i} \gives \NE_i,\BB_i,\WW_i \\
\NE' = \C{i=1}{n}{\NE_i},~ \BB' = \C{i=1}{n}{\BB_i},~ \WW' = \C{i=1}{n}{\WW_i}}
{\TE,\EE,\NE,\BB,\WW \vdash \bindings{npat}{nexpr}{n} \gives \NE \oplus \NE', \BB \oplus \BB', \WW \oplus
  \WW'}

\ruleheader{\TE,\EE,\NE,\BB,\WW \vdash \binding{npat}{nexpr} \gives \NE',\BB',\WW'}

\infrule[NetBinding]
{\TE,\EE,\NE \vdash \emtxt{nexp} \gives \vv,\BB',\WW' \andalso \NE,\BB' \vdash_n \emtxt{npat}, \vv \gives \NE',\WW''}
{\TE,\EE,\NE,\BB,\WW \vdash \emtxt{npat}~\tm{=}~\emtxt{nexp} \gives \NE \oplus \NE',~ \BB \oplus
  \BB',~ \WW\oplus\WW'\oplus\WW''}

Each definition potentially augments the variable environment and the box and wire sets.
Evaluating an expression potentially creates boxes and wires. Binding a pattern may only create
wires.

\medskip
Network-level pattern binding is handled using the following rules, where

\begin{center}
$\NE,\BB \vdash_n \emtxt{npat}, \rho \gives \NE', \WW'$
\end{center}

means that in the context of $\NE$ and $\BB$ binding \emph{npat} to value $\rho$ results in a 
network-level environment $\NE$ and a set of wires $\WW'$.

\ruleheader{\NE,\BB \vdash_n \text{NPat},\rho \gives \NE',\WW}

\infrule[TNPatVar]
{}
{\NE,\BB \vdash_n \id, \rho \gives [\id \mapsto \rho],~\emptyenv}

\infrule[TNPatTuple]
{\forallin{i}{1}{n}, ~~\NE,\BB \vdash_n \emtxt{npat}_i, \rho_i \gives \NE_i,~\WW_i}
{\NE,\BB \vdash_n \tm{(}\emtxt{npat}_1,\ldots,\emtxt{npat}_n\tm{)}, \tuplen{\rho} \gives
  \C{i=1}{n}{\NE_i},~\C{i=1}{n}{\WW_i}}

\infrule[TNPatOutput]
{\NE(\id)=\sem{Loc}(l',1) \andalso \rho=\sem{Loc}(l,s) \andalso
  \BB(l)=\langle\sem{BOut},\ldots\rangle \andalso k=\textsc{NewWid}()}
{\NE,\BB \vdash_n \id, \rho \gives \emptyenv,~ [k\mapsto \ttuple{\ttuple{l}{s}}{\ttuple{l'}{1}}]}

The last rule handes the case where a an identifier previously declared as a stream output is bound
to an expression. In this case, a wire is inserted, connecting the box resulting from the evaluation
of this expression to the box which has been created when instanciating the stream output.

The function $\textsc{NewWid}$ returns a new, fresh wire index (i.e. an index $k$ such as $k \not\in Dom(\WW)$).


\subsection{Network expressions}
\label{sec:static-network-expressions}

Here the rules are fairly standard, except for those dealing with instanciation of actors and
recursive \texttt{let} definitions.

\ruleheader{\TE,\EE,\NE \vdash \text{NExp} \gives \rho,\BB,\WW}

% \infrule[NAct]
% {\NE(\id) =
%   \sem{Act}~\tttttttuple{\emtxt{id}}{\tuplek{\emtxt{id}}{p}}{\emtxt{ins}}{\emtxt{outs}}{\emtxt{vars}}{\emtxt{rules}} \\
%  \forallin{i}{1}{n}, ~~\TE,\EE \vdash \emtxt{exp}_i \gives \vv_i
%  \andalso \emtxt{is\_static\_const}(v_i)}
% {\TE,\EE,\NE \vdash \id ~\tuplen{\emtxt{exp}} \gives
%   \sem{Act}~\tttttuple{\id}{\tuplek{\emtxt{id}}{k}}{\tuplek{\vv}{k}}{\emtxt{ins}}{\emtxt{outs}},~ \emptyenv,~ \emptyenv}

% \infrule[NVar]
% {\NE(\id) = \rho \andalso \rho \not= \sem{Act}~\tttttuple{.}{.}{.}{.}{.}}
% {\TE,\EE,\NE \vdash \id ~\emptytuple \gives \rho,~ \emptyenv,~ \emptyenv}

% The first rule refers to variable bound to actors (with a possible set of actual parameter
% values). The second with all other variables. All variables are looked up in the network-level
% environment. Evaluations of actor parameters may refer to variables defined at the expression-level.

\infrule[NNVar]
{\NE(\id) = \rho}
{\TE,\EE,\NE \vdash \id \gives \rho,~ \emptyenv,~ \emptyenv}

\infrule[NEVar]
{\EE(\id) = \vv,~~ \emtxt{wrapable}(\vv)}
{\TE,\EE,\NE \vdash \id \gives \sem{NEVal}(\vv) ,~ \emptyenv,~ \emptyenv}

The value of an identifier appearing in a network expression is searched first in the network-level
environment and, if this fails, in the expression-level environment. The predicate \emph{wrapable}
indicates whether an expression-level value $\vv$ can be wrapped as network-level value (in other
words whether this value can be used as a parameter value for an actor). It is only true for
$\sem{Int}$ and $\sem{Bool}$ constants.

\infrule[NEConst]
{}
{\TE,\EE \vdash \text{int}/\text{bool} \gives \sem{NEVal}(\sem{Int}/\sem{Bool})}

\infrule[NTuple]
{\forallin{i}{1}{n}, ~~\NE \vdash \emtxt{nexp}_i \gives \rho_i, \BB_i, \WW_i}
{\TE,\EE,\NE \vdash \tm{(}\emtxt{nexp}_1,\ldots,\emtxt{nexp}_n\tm{)} \gives \tuplen{v},~ \C{i=1}{n}{\BB_i},~ \U{i=1}{n}{\WW_i}}

\infrule[NFun]
{}
{\TE,\EE,\NE \vdash \tm{function}~\emtxt{npat}~\tm{\rightarrow}~\emtxt{nexp}~\gives
  \sem{Clos}(\emtxt{npat},\emtxt{nexp},\NE),~ \emptyenv,~ \emptyenv}

\infrule[NLetDef]
{\TE,\EE,\NE,\emptyset,\emptyset \vdash \bindings{npat}{nexp}{n} \gives \NE', \BB, \WW \\
 \TE,\EE,\NE \oplus \NE' \vdash \emtxt{nexp}_2 \gives \rho, \BB', \WW''}
{\TE,\EE,\NE \vdash \tm{let}~ \bindings{npat}{nexp}{n}~ \tm{in}~ \emtxt{nexp}_2 \gives \rho,~ \BB,~ \WW}

The above rule is for \emph{non recursive} definitions. Recursive definitions are handled
in Sec.~\ref{sec:recurs-netw-decl}.

% \infrule[Match]
% {\NE \vdash \emtxt{nexp} \gives \rho, \BB, \WW \andalso \existsin{j}{1}{n},~~\vdash_p \emtxt{npat}_j, \rho \gives \NE_j \\
% \NE \oplus \NE_j \vdash \emtxt{nexp}_j \gives \rho', \BB', \WW'}
% {\NE \vdash
%   \tm{match}~\emtxt{nexp}~\tm{with}~(\emtxt{nexp}_1~\tm{\rightarrow}~\emtxt{nexp}_1)~\ldots~(\emtxt{npat}_n~\tm{\rightarrow}
%  ~\emtxt{nexp}_n)
%   \gives \rho',~ \BB\oplus\BB',~ \WW\cup\WW'}

% \note{Strictly speaking, the \textsc{Match} rule is non-deterministic since it does not specify what happens
% if several cases of a \texttt{match} expression are valid -- \emph{i.e.} if there exits several $j$
% such that $\vdash_p \emtxt{npat}_j, \rho \gives \EE_j$ holds. TO BE DISCUSNED ?}

% \infrule[If True]
% {\NE \vdash \emtxt{nexp} \gives \truee,\BB,\WW \andalso \NE \vdash \emtxt{nexp}_1 \gives \rho,\BB',\WW'}
% {\NE \vdash \tm{if}~ \emtxt{nexp}~ \tm{then}~ \emtxt{nexp}_1~ \tm{else}~ \emtxt{nexp}_2 \gives
%   \rho,\BB\oplus\BB',\WW\cup\WW'}

% \infrule[If False]
% {\NE \vdash \emtxt{nexp} \gives \falsee,\BB,\WW \andalso \NE \vdash \emtxt{nexp}_2 \gives \rho,\BB',\WW'}
% {\NE \vdash \ifff \emtxt{nexp}~ \tm{then}~ \emtxt{nexp}_1~ \tm{else}~ \emtxt{nexp}_2 \gives \rho,\BB\oplus\BB',\WW\cup\WW'}


\todo{Nil,Cons,If,Match,Let}

% \infrule[Let Rec]
% {\Sigma,\EE,\GG \vdash \lett \recc \emtxt{npat} \eqq \expp_2 \gives \EE',\GG'
%   \andalso \Sigma, E \oplus \EE',\GG' \vdash \expp_1 \gives \rho', \GG''}
% {\Sigma,\EE,\GG \vdash \lett \recc \emtxt{npat} \eqq \expp_2 \inn \expp_1 \gives \rho', \GG''}

% \medskip
% Pattern binding at the expression level is handled using the following rules, where $\vdash_p \emtxt{npat}, \rho
% \gives \NE$ means that binding \emph{npat} to value $\rho$ results in a 
% static environment $\NE$ :

% \ruleheader{\vdash_p \text{NPat},\rho \gives \NE}

% \infrule[NPat Id]
% {}
% {\vdash_p \id, \rho \gives [\id \mapsto \rho]}

% \infrule[NPat Ignore]
% {}
% {\vdash_p \tm{\_}, \rho \gives \emptyenv}

% \infrule[NPat Tuple]
% {\forallin{i}{1}{n}, ~~\vdash_p \emtxt{npat}_i, \rho_i \gives \NE_i}
% {\vdash_p \tm{(}\emtxt{npat}_1,\ldots,\emtxt{npat}_n\tm{)}, \tuplen{\rho} \gives
%   \C{i=1}{n}{\NE_i}}

\bigskip
The following rules deals with applications.

\medskip
Rule \textsc{NAppClo} deals with the application of closures and follows the classical call-by-value
strategy (the closure body is evaluated in an environment augmented with the bindings resulting
from binding its pattern to the argument). 

\infrule[NAppClo]
{\TE,\EE,\NE \vdash \emtxt{nexp}_1 \gives \Clos(\emtxt{npat},\emtxt{nexp},\NE'), \BB, \WW \\
 \TE,\EE,\NE \vdash \emtxt{nexp}_2 \gives \rho,\BB',\WW' \\
 \NE,\BB \vdash_n \emtxt{npat},\rho \gives \NE'', . \\
 \NE' \oplus \NE'' \vdash \emtxt{nexp} \gives \rho', \BB'', \WW''}
{\TE,\EE,\NE \vdash \emtxt{nexp}_1 ~\emtxt{nexp}_2 \gives \rho',\BB\oplus\BB'\oplus\BB',\WW\cup\WW'\cup\WW''}

% Rule \textsc{App Prim} deals with the application of builtin primitives (\texttt{+}, \texttt{-},
% \ldots). For each $\prim \in \Prim$ we assume that $\prim(v_1,\dots,v_k)$ denotes the proper
% resulting value, provided that $k$ is the proper number of arguments for operator $\prim$ and
% that the arguments $v_1,\dots,v_k$ are of the proper type\footnote{This is ensured by the
%   the type cheking stage.}.

% \infrule[App Prim]
% {\NE \vdash \emtxt{nexp}_1 \gives \Prim(p), \BB,\WW \\
%  \NE \vdash \emtxt{nexp}_2 \gives \rho,\BB',\WW' \andalso p(\rho)=\rho'}
% {\NE \vdash \emtxt{nexp}_1 ~\emtxt{nexp}_2 \gives \rho',\BB\oplus\BB', \WW\oplus\WW'}

\medskip Rules \textsc{NAppAct1} and \textsc{NAppAct2} deals with the instanciation of actors. Both
insert a new box and a set of new wires connecting the argument to the inputs of the inserted
box. The former deals with actors with parameters, the latter with actors without parameters.

\infrule[NAppAct1]
{\TE,\EE,\NE \vdash \emtxt{nexp}_1 \gives \\
  \sem{Act}\ttttttuple{\emtxt{id}}{\tuplek{p}{q}}{\tuplek{i}{m}}{\tuplek{o}{n}}{\mapset{\emtxt{v}}{\emtxt{exp}}{k}}{\emtxt{rules}},~~ \BB', \WW' \\
 \forallin{i}{1}{k},~ \TE,~\EE \oplus \C{i=1}{q}{[\emtxt{p}_i \mapsto \vv'_i]} \vdash \emtxt{exp}_i \gives \vv_i \\
 TE,\EE,NE \vdash \emtxt{nexp}_2 \gives \langle\sem{NEVal}(\vv'_1),\ldots,\sem{NEVal}(\vv'_q)\rangle, \BB',\WW' \\
 TE,\EE,NE \vdash \emtxt{nexp}_3 \gives \langle\Loc(l_1,s_1),\ldots,\Loc(l_m,s_m)\rangle, \BB'',\WW'' \\
 \txt{b} = \tttttttuple{\emtxt{id}}{\sem{BBox}}{\mapset{\emtxt{p}}{\vv'}{q}}{\mapset{\emtxt{v}}{\vv}{k}}{\tuplek{\ut{bi}}{m}}{\tuplek{\ut{bo}}{n}}{\emtxt{rules}} \\
 l = \textsc{NewBid}() \\
 \forallin{j}{1}{m},~~ k_j=\textsc{NewWid}(),~~\txt{w}_j = \sem{Wire}\ttuple{\ttuple{l_j}{s_j}}{\ttuple{l}{j}} \\
 \forallin{j}{1}{m},~~\ut{bi}_j=\ttuple{i_j}{0} \\
 \forallin{j}{1}{n},~~\ut{bo}_j=\ttuple{o_j}{0} \\
 \WW''' = \{k_1 \mapsto \txt{w}_1,\ldots,k_m \mapsto \txt{w}_m\}}
{\TE,\EE,\NE \vdash \emtxt{nexp}_1 ~\emtxt{nexp}_2 ~\emtxt{nexp}_3 \\
\gives \langle\Loc(l,1),\ldots,\Loc(l,n)\rangle,~~ [l \mapsto \txt{b}] \oplus \BB'' \oplus \BB' \oplus \BB,~~ \WW''' \cup \WW'' \cup \WW' \cup \WW}

\infrule[NAppAct2]
{\TE,\EE,\NE \vdash \emtxt{nexp}_1 \gives \\
  \sem{Act}\ttttttuple{\emtxt{id}}{\emptytuple}{\tuplek{i}{m}}{\tuplek{o}{n}}{\mapset{\emtxt{v}}{\emtxt{exp}}{k}}{\emtxt{rules}},~~ \BB', \WW' \\
 \forallin{i}{1}{k},~ \TE,\EE \vdash \emtxt{exp}_i \gives \vv_i \\
 TE,\EE,NE \vdash \emtxt{nexp}_2 \gives \langle\Loc(l_1,s_1),\ldots,\Loc(l_m,s_m)\rangle, \BB'',\WW'' \\
 \txt{b} =
 \tttttttuple{\emtxt{id}}{\sem{BBox}}{[]}{\mapset{\emtxt{v}}{\vv}{k}}{\tuplek{\ut{bi}}{m}}{\tuplek{\ut{bo}}{n}}{\emtxt{rules}}
 \andalso l = \textsc{NewBid}() \\
 \forallin{j}{1}{m},~~ k_j=\textsc{NewWid}(),~~\txt{w}_j = \sem{Wire}\ttuple{\ttuple{l_j}{s_j}}{\ttuple{l}{j}} \\
 \forallin{j}{1}{m},~~\ut{bi}_j=\ttuple{i_j}{0} \\
 \forallin{j}{1}{n},~~\ut{bo}_j=\ttuple{o_j}{0} \\
 \WW''' = \{k_1 \mapsto \txt{w}_1,\ldots,k_m \mapsto \txt{w}_m\}}
{\TE,\EE,\NE \vdash \emtxt{nexp}_1 ~\emtxt{nexp}_2 \\
 \gives \langle\Loc(l,1),\ldots,\Loc(l,n)\rangle,~~ [l \mapsto \txt{b}] \oplus \BB'' \oplus \BB' \oplus \BB,~~ \WW''' \cup \WW'' \cup \WW' \cup \WW}

\subsection{Recursive network definitions }
\label{sec:recurs-netw-decl}

Recursive definitions may appear both within network expressions or at the network declaration level.

\ruleheader{\TE,\EE,\NE \vdash \text{NExp} \gives \rho,\BB,\WW}

\infrule[NLetRecDef]
{\TE,\EE,\NE,\emptyset,\emptyset \vdash \bindings{npat}{nexp}{n}_{rec} \gives \NE', \BB, \WW \\
 \TE,\EE,\NE \oplus \NE' \vdash \emtxt{nexp}_2 \gives \rho, \BB', \WW''}
{\TE,\EE,\NE \vdash \tm{let}~ \tm{rec}~ \bindings{npat}{nexp}{n}~ \tm{in}~ \emtxt{nexp}_2 \gives \rho,~ \BB,~ \WW}

\ruleheader{\TE,\EE,\NE,\BB,\WW \vdash \text{NetRecDecl} \gives \NE',\BB',\WW'}

\infrule[NetRecDecl]
{\TE,\EE,\NE,\emptyset,\emptyset \vdash \bindings{npat}{nexp}{n}_{rec} \gives \NE', \BB', \WW'}
{\TE,\EE,\NE,\BB,\WW \vdash \tm{net}~ \tm{rec}~ \bindings{npat}{nexp}{n} \gives \NE \oplus \NE', \BB \oplus \BB', \WW \oplus
  \WW'}

Both cases are handled with a common rule \textsc{NetRecBindings}.
This rule supports only two kinds of recursive bindings :
\begin{itemize}
\item all the bound identifiers are \emph{functions},
\item all the bound identifiers \emph{wires}.
\end{itemize}

In the former case, the result is a circular closure (or a set of mutually recursive closures in
case of multiple bindings).

In the latter case, the recursively defined values correspond to \emph{cycles} in the network.

\subsubsection{Recursive functions}
\label{sec:recursive-functions}

% \infrule[NetBindings]
% {\forallin{i}{1}{n},~~ \TE,\EE,\NE,\BB,\WW \vdash \ibinding{npat}{nexp}{i} \gives \NE_i,\BB_i,\WW_i \\
% \NE' = \C{i=1}{n}{\NE_i},~ \BB' = \C{i=1}{n}{\BB_i},~ \WW' = \C{i=1}{n}{\WW_i}}
% {\TE,\EE,\NE,\BB,\WW \vdash \bindings{npat}{nexpr}{n} \gives \NE \oplus \NE', \BB \oplus \BB', \WW \oplus
%   \WW'}

% \infrule[NetRecBindings]
% {\forallin{i}{1}{n},~~ \TE,\EE,\NE,\BB,\WW \vdash \emtxt{nbind} \gives \NE_i,\BB_i,\WW_i \\
% \NE' = \C{i=1}{n}{\NE_i},~ \BB' = \C{i=1}{n}{\BB_i},~ \WW' = \C{i=1}{n}{\WW_i}}
% {\TE,\EE,\NE,\BB,\WW \vdash \tm{net}~ \tm{rec}~ \tm{rec} \bindings{npat}{nexp}{n} \gives \NE \oplus \NE', \BB \oplus \BB', \WW \oplus
%   \WW'}

\ruleheader{\TE,\EE,\NE,\BB,\WW \vdash \bindings{npat}{nexpr}{n}_{rec} \gives \NE',\BB',\WW'}

\infrule[NRecBindingsF]
{\forallin{i}{1}{n},~~ \NE_i=[\txt{id} \mapsto \txt{cl}_i]\\
 \forallin{i}{1}{n},~~ \txt{cl}_i = \sem{Clos}(\emtxt{npat}_i,\emtxt{nexp}_i,\NE \oplus \NE')]\\
\NE' = \C{i=1}{n}{\NE_i}}
{\TE,\EE,\NE \vdash \langle~ \txt{id}_1 \tm{=}~
  \tm{function}~\emtxt{npat}_1~\tm{\rightarrow}~\emtxt{nexp}_1, \ldots, \txt{id}_n \tm{=}~
  \tm{function}~\emtxt{npat}_n~\tm{\rightarrow}~\emtxt{nexp}_n\rangle_{rec}~\\
 \gives \NE', \emptyenv, \emptyenv}

% \infrule[NRecDecls1]
% {\forallin{i}{1}{n},~~ \NE_i=[\txt{id} \mapsto \sem{Clos}(\emtxt{npat}_i,\emtxt{nexp}_i,\NE \oplus \NE')]\\
% \NE' = \C{i=1}{n}{\NE_i}}
% {\TE,\EE,\NE \vdash \tm{let}~ \tm{rec}~ \langle~ \txt{id}_1 \tm{=}~
%   \tm{function}~\emtxt{npat}_1~\tm{\rightarrow}~\emtxt{nexp}_1, \ldots, \txt{id}_n \tm{=}~
%   \tm{function}~\emtxt{npat}_n~\tm{\rightarrow}~\emtxt{nexp}_n\rangle~\\
%  \gives \NE', \emptyenv, \emptyenv}

\subsubsection{Recursive wires}
\label{sec:recursive-wires}

If the defined value is \emph{not} a function, then the recursively defined values
correspond to \emph{cycles} in the network. Evaluation is then carried out as follows:

\begin{enumerate}
\item First, we create a \emph{recursive environment} $\NE'$, by binding each identifier occuring in
  the LHS patterns to the location of a temporary, freshly created, box with tag $\sem{BDummy}$. This
  can be formalized with the following set of rules:

\ruleheader{\vdash_r \text{NPat} \gives \NE,\BB}

\infrule[]
{\text{b}=\langle\txt{id},\sem{BDummy},\emptytuple,\tuple{"i" \mapsto 0},\tuple{"o"\mapsto 0}\rangle \andalso
  \txt{l}=\textsc{NewBid}()}
{\vdash_r \id \gives [\id \mapsto \sem{Loc}(l,0)],~ [l \mapsto \txt{b}]}

\infrule[]
{\forallin{i}{1}{n}, ~~\vdash_r \emtxt{npat}_i \gives \NE_i,~\BB_i}
{\vdash_r \tm{(}\emtxt{npat}_1,\ldots,\emtxt{npat}_n\tm{)} \gives \C{i=1}{n}{\NE_i},~\C{i=1}{n}{\BB_i}}

\item Second, all the RHS expressions are evaluated in an environment augmented with $\NE'$. The
resulting values are bound to the LHS patterns using the usual rules defined in
section~\ref{sec:static-network-declarations}, leading an environment $\NE''$.

\item Third, for each identifier $r$ occuring in the recursive environment $\NE'$, we build a
\emph{substitution} $\phi = \lbrace u \mapsto v \rbrace$, where $u=\NE'(r)=Loc(l,s)$ and
$v=\NE''(r)=Loc(l',s')$. This results in a set $\Phi$ of substitutions.

\item Fourth, we apply each $\phi \in \Phi$ to the set $\WW'$ of wires produced by the evaluation of
  the expressions evaluated at step 2. More precisely, we replace each pair $\ttuple{\gamma}{\gamma'} \in \WW'$ by
$\ttuple{\phi(\gamma)}{\phi(\gamma')}$, where $\phi(\gamma)$ is defined as follows:

if $\phi = \lbrace \Loc(k,s) \mapsto \Loc(k',s') \rbrace$ and $\gamma=\Loc(l,s'')$ then
\begin{equation}
\phi(\gamma) =
\begin{cases}
\Loc(k',s') & \text{if } l=k, \\
\gamma & \text{otherwise}.
\end{cases}
\end{equation}
\end{enumerate}

The process can be summarized in the following rule:

\infrule[NRecBindingsV]
{\forallin{i}{1}{n},~ \vdash_r \emtxt{npat}_i \gives \NE'_i, \BB_i \\
  \NE' = \C{i=1}{n}{\NE'_i} \andalso \BB = \C{i=1}{n}{\BB_i} \\
 \forallin{i}{1}{n},~ \TE,\EE,\NE'\oplus\NE \vdash \emtxt{nexp}_i \gives \rho_i, \BB'_i, \WW'_i \\
 \forallin{i}{1}{n},~ \NE, \BB \vdash_n \emtxt{npat}_i, \rho_i \gives \NE''_i, \WW'''_i \\
  \NE'' = \C{i=1}{n}{\NE''_i} \\
 \Phi = \lbrace \NE'(r) \mapsto \NE''(r), ~\forall r \in dom(\NE') \rbrace \\
 \WW'' = \Pi_{\Phi} (\WW')}
{\TE,\EE,\NE \vdash \bindings{npat}{nexp}{n}_{rec} \gives \NE'', \BB', \WW''}

% \infrule[NLetRecDef]
% {E \oplus \EE',\GG \vdash \expp \gives \vv,\GG' \andalso \vdash \patt, \vv \gives \EE'}
% {\EE,\GG \vdash \lett \recc \patt \eqq \expp \gives \EE \oplus \EE', \GG'}

%%% Local Variables: 
%%% mode: latex
%%% TeX-master: "caph"
%%% End: 
