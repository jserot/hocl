%%%%%%%%%%%%%%%%%%%%%%%%%%%%%%%%%%%%%%%%%%%%%%%%%%%%%%%%%%%%%%%%%%%%%%%%%%%%%%%%%%%%%%%%%
%%                                                                                     %%
%%                This file is part of the HoCL Compiler distribution                  %%
%%                                                                                     %%
%%      Copyright 2019 Jocelyn SEROT (jocelyn.serot@uca.fr).  All rights reserved.     %%
%%         This file is distributed under the terms of the MIT Library License         %%
%%                                                                                     %%
%%%%%%%%%%%%%%%%%%%%%%%%%%%%%%%%%%%%%%%%%%%%%%%%%%%%%%%%%%%%%%%%%%%%%%%%%%%%%%%%%%%%%%%%%

\chapter{Introduction}
\label{chap:intro}

This document is a short introduction to the \hocl programming language. \hocl is ... TBW

\medskip This report is structured as follows:  TBW

% the remainder of this chapter provides motivation and
% general background; Chapter~\ref{chap:overview} is an overview of the \caph language design,
% including informal descriptions of the expression, network and actor sub-languages. The full
% concrete syntax is given in chapter~\ref{chap:syntax} and the abstract syntax of the core language
% in chapter~\ref{chap:abssyn}. Chapter~\ref{chap:typing} gives the formal typing rules for \caph
% programs. Chapter~\ref{chap:static} gives the formal static semantics,
% \emph{i.e.} the interpretation of \caph programs as \emph{data-flow graphs}.
% Chapter~\ref{chap:dynamic-semantics} gives the formal dynamic semantics, which provides a way to
% \emph{simulate} \caph programs.  Chapter~\ref{chap:interm-repr} describes
%   the intermediate representation of \caph programs used as an input for the back-end code
%   generators. Chapter~\ref{cha:using-caph-compiler} describes, pragmatically, how to use the
% compiler in order to obtain graphical representations of programs, simulate them or invoke the
% SystemC or VHDL backend in order to generate code. Chapter~\ref{cha:stdlib} gives the contents of
% some ``standard libraries''. Chapter~\ref{cha:foreign-funct-interf} describes
% the mechanism by which existing C or VHDL functions can be used by \caph
% programs. Chapter~\ref{cha:compiler-options} is a summary of compiler options.
%  Chapter~\ref{cha:variants-impl} is a short, technical, overview of how \emph{variant
%   types} (described in chapter~\ref{chap:overview}) are implemented in SystemC and VHDL. 

\clearpage

\section{Motivation, goals and principles}

TBW

\begin{itemize}
\item describing dataflow process networks in an implementation-independant way (s/w, h/w, both...)
\item replace tedious GUI-based specification by a purely textual description, amenable to
  \begin{itemize}
  \item verification (by means of type-checking)
  \item abstraction
  \item versionning
  \item \ldots
  \end{itemize}
\item does not deal with the specification of individual actor behavior (actors are viewed as black
  boxes\footnote{Except for their prod-cons rates in the case of SDF actors}
\item backends for specific implementation-specific tools (ex: \textsc{preesm})
\item can be viewed as a special case of a \emph{coordination language} taking advantage of the
  natural duality between dataflow networks and functional programming languages
\end{itemize}


%%% Local Variables: 
%%% mode: latex
%%% TeX-master: "hocl"
%%% End: 
